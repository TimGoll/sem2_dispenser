% !TeX encoding = UTF-8
% !TeX root = ../Vitamintablettenspender.tex

\chapter{Konzeptionierung}

Im Entwicklungsprozess nimmt die Suche nach der optimalen Lösung für das vom Kunden gewünschte Produkt die Hauptaufgabe ein. Das Ergebnis soll nachvollziehbar und objektiv bewertbar sein. Dafür wird im Folgenden zunächst eine umfassende Planung des Produktes hinsichtlich Markt- und Wettbewerbschancen durchgeführt.

Die Entwicklung des Konzepts erfolgt darauffolgend anhand eines Projektplans, der den vom Kunden gewünschten Termin zur Vorstellung des Produktes mit einem Prototypen berücksichtigt. In der Anforderungsliste werden die Anforderungen des Kunden aus dem Lastenheft konkretisiert und durch interne Spezifikationen ergänzt. Damit wird eine Basis zur Entwicklung von Lösungsideen geschaffen.

Hierfür werden zunächst die Zusammenhänge von Anforderungen und Funktionen abstrakt in der Funktionsstruktur dargestellt, wobei das Loslösen vom Gegenständlichen und von vorzeitigen Festlegungen auf ein bestimmtes Lösungskonzept ermöglicht wird. Zur Systematisierung der Suche und Auswahl des optimalen Lösungsprinzips wird im morphologischen Kasten alle Lösungsoptionen aller Teilfunktionen berücksichtigt und verschiedene Gesamtlösungskombinationen unter Beachtung von Konflikten untereinander gebildet. Die abgesicherte Festlegung des zu realisierenden Lösungskonzeptes erfolgt in der Nutzwertanalyse. Zuletzt wird ein Grobentwurf zur Verdeutlichung des Wirkprinzips angefertigt.

\section{Produktplanung}
Der Markt für Nahrungsergänzungsmittel ist riesig. Mehrere Studien und Umfragen zeigten bereits die enorme Nachfrage deutschlandweit. So wurden nach einer Studie von Insight Health \cite{studie1}, auf der Website des deutschen Lebensmittelverbandes veröffentlicht, dass im Jahr 2018 225 Millionen Packungen Nahrungsergänzungsmittel verkauft wurden. Das entspricht einem Umsatz von circa 1,439 Milliarden Euro. Dabei machen Vitamine und Mineralstoffe näherungsweise zwei Drittel des gesamten Nahrungsergänzungsmittel-Marktes aus. Das Vitamin C, zur Stärkung des Immunsystems, ist dabei in der Sparte der Vitaminprodukte mit großem Abstand am beliebtesten, sodass davon im Jahr 2019 29,2 Millionen Packungen abgesetzt wurden. Zweitplatziert sind Multivitamin-Tabletten mit Mineralstoffen, die immernoch 17,5 Millionen verkaufte Packungen vorweisen können. Aber auch Vitamintabletten, die lediglich das Vitamin B12 oder die Vitamine A/D oder B-Komplexe beinhalten, sind sehr stark gefragt. Aber auch unabhängig von Vitaminen werden einige Mineralien wie Magnesium insbesondere für Sportler aber auch Calcium vom Markt aufgekauft. Dabei wurden im Jahr 2018 36,8 Millionen Packungen Magnesium sowie 16,4 Millionen Packungen Calcium verkauft. Die Chancen, dass ein Produkt zur automatisierten Ausgabe von Tabletten mit Vitaminen und Mineralen vertrieben wird, sind damit gegeben. Die bereits große Akzeptanz von Medikamentendispensern stellen ein gutes Beispiel für die Erfolgsaussichten des zu entwickelnden Produktes. Diese Dispenser verringern die Wahrscheinlichkeit falsche Medikamente einzunehmen und bieten dabei zeitgleich einen hohen Komfort. Mit dem Vitamintablettendispenser wird ein Produkt konstruiert, das der breiten Masse der Bevölkerung zugänglich gemacht werden kann. Aufgrund der hohen Verkaufszahlen bieten sich gute Chancen durch den erhöhten Komfort der Tablettenzubereitung und der damit erleichterten Tabletteneinnahme.\\
Die Technologie zum Realisieren des Produktwunsches sind gegeben. Es gibt zahlreiche Möglichkeiten zum Entwerfen eines Ausgabemechanismus von Vitamintabletten. Die vielfältigen Motoren zum Erzeugen des gewünschten Schiebeprinzips. Damit werden zur Umsetzung ausschließlich Komponenten Technologien aus dem Bereich der Basistechnologien sowie der Alten Technologien verwendet. Diese sind preislich günstig zu bekommen und können durch geschickte Kombination miteinander Gewinne einbringen. Hierfür wurden in Norwegen im Jahr 2018 elektronische Medikamentendispenser in Krankenhäuser in einem Pilotprojekt getestet \cite{studie2}. Damit sind die technischen Voraussetzungen für das Produkt gegeben.\\
Im direkten Wettbewerb mit dem automatischen Vitamintablettendispenser steht ein Produkt der Firma tespo, das die automatische Ausgabe ihrer firmeneigenen Vitamine in Pulverform. Des Weiteren gibt es ausschließlich Behältnisse zu Aufbewahrung von Tabletten. Diese stellen jedoch nicht die automatische Ausgabe sowie das anschließende Auffüllen mit Wasser bereit. Die gewünschte Bedienbarkeit mit einem Touchdisplay sind zusammen mit der automatischen Zubereitung Alleinstellungsmerkmale. Damit ist das Produkt ein Unikat und hebt sich gegenüber der Konkurrenz deutlich ab.\\
In der Patentanalyse wurde mittels einer Bottom-Up-Recherche-Methode sichergestellt, dass es keine vergleichbaren Patente des Produkts gibt. Dabei wurden im Depatisnet, Espacenet sowie unter Google Patents keine relevanten Produkte gefunden. Es existiert ausschließlich das deutsche Patent DE000029811862U1, der Vitamin-Tablettenspender. Das im Jahr 1998 angemeldete Gebrauchsmuster beschreibt einen Spender, an dem Tablettenröhrchen befestigt werden können Mit Hilfe eines mechanischen Schiebers können einzelne Tabletten herausgenommen werden. Das Einzelelement ist durch eine Wandhalterung gekennzeichnet, an der die Röhrchen einzeln ausgetauscht werden können. Das Patent beinhaltet das Funktionsprinzip des zu entwickelnden Produktes nur eingeschränkt. Das Patent schützt eine komplette Wandhalterung durch Verschraubung von den Produktschenkeln, das direkt Röhrchen montiert werden können, sowie den mechanischen Schiebemechanismus. Das zu entwickelnde Produkt erweitert das Prinzip in seiner Funktion, sodass der Wirkbereich des Schutzes verlassen wird. Der automatische Prozess zur Tablettenausgabe löst dabei den Schiebemechanismus ab. Das System ist nicht zur Montage an der Wand vorgesehen. Außerdem ist die Befüllung des Gerätes mit den Tabletten angedacht, sodass die einzelnen Röhrchen nicht direkt am Produkt montiert werden. Durch Hinzufügen von wesentlichen Produktfunktionen wie beispielsweise einem Touchscreen, einem Wasserbehälter zum automatischen Auffüllen des Behälters mit Wasser und einigen weiteren Zusatzfunktionen hebt sich das System deutlich vom Patent ab und kann konstruiert werden. Die Kombination der einzelnen Komponenten für sein Anwendungsgebiet kann als Patent angemeldet werden. Ausschließlich der automatisierte Auswurf muss hinsichtlich seiner Ähnlichkeit zum Schiebemechanismus überprüft und abgesichert werden.\\
Evtl. kurze Produktanalyse...

\section{Projektplan}

Für die Gestaltungsaufgabe mit Planung von

\begin{itemize}
\item Aufgaben
\item Dauer, Beginn und Ende der Aufgaben
\item Abhängigkeiten zwischen Aufgaben
\item ggf. kritischem Pfad
\end{itemize}

\section{Anforderungsliste}

\begin{itemize}
	\item Anforderungen des Lastenheftes präzisiert sowie um sinnvolle Anforderungen mund Angaben inkl. Verweis auf Lage im Kano-Diagramm ergänzt (min. 20 Anforderungen)
	\item Konsistenzmatrix für (min. 10) Hauptanforderungen
\end{itemize}

\newcolumntype{L}[1]{>{\raggedright\arraybackslash}p{#1}} % linksbündig mit Breitenangabe
\newcolumntype{C}[1]{>{\centering\arraybackslash}p{#1}} % zentriert mit Breitenangabe
\newcolumntype{R}[1]{>{\raggedleft\arraybackslash}p{#1}} % rechtsbündig mit Breitenangabe

\begin{longtable}{C{0.05\linewidth}C{0.05\linewidth}C{0.05\linewidth}L{0.75\linewidth}}
	\toprule
 	
 	\textbf{Nr.} & \textbf{F/W} & \textbf{Gew.} & \textbf{Beschreibung und Erläuterung}  \\
	
	\toprule
	\endfirsthead
	
	\textbf{1} & & & \textbf{Funktionsanforderungen}  \\
	1.1 & F & & Wirft gewünschte Vitamintablette aus  \\
	1.2 & W & & Tabletten werden automatisch ausgeworfen, wenn Glas platziert wird \\
	1.3 & W & & Immer nur eine der Tageszeit entsprechende Tablette darf ausgeworfen werden um Vitaminbalance zu garantieren \\
	1.4 & W & & Modernes Design \\
	1.5 & F & & Gute Standfestigkeit oder Möglichkeit zu Wandmontage \\
	1.6 & F & & Anzeige der aktuellen Uhrzeit \\
	1.7 & F & & Anzeige der nächsten Tablette mit deren Uhrzeit \\
	1.8 & F & & Einfache und intuitive Bedienung \\
	1.9 & W & & Nutzerprofile, sodass zwischen mehreren Nutzern gewechselt werden kann \\
	1.10 & F & & Vorrat-Leer Sensor um Nutzer auf Röhrchenwechsel hinzuweisen \\
	1.11 & W & & Warnton, wenn Tablette vergessen wurde \\
	
	\midrule
	
	\textbf{2} & & & \textbf{Mechanische/Geometrische Anforderungen} \\
	2.1 & F & & Muss kompatibel sein mit handelsüblichen Vitaminröhrchen \\
	2.2 & F & & Verwendung robuster Sensoren um Lebensdauer zu erhöhen \\
	2.3 & W & & Handliches Gewicht: $<\,5\,\text{kg}$ \\
	2.4 & W & & Maximale Abmessungen: $X\,\text{cm}\,\times\,Y\,\text{cm}\,\times\,Z\,\text{cm}$ \\
	
	\midrule
	
	\textbf{3} & & & \textbf{Sicherheitsanforderungen} \\
	3.1 & F & & Auswurfsmechanismus muss geschützt sein, sodass Finger nicht versehentlich hinein gerät \\
	3.2 & F & & Standfeste Position des Glases \\
	
	\midrule 
	
	\textbf{4} & & & \textbf{Umwelt- und Wartungsanforderungen} \\
	4.1 & F & & Verwendung von lebensmittelechten Materialien \\
	
	\midrule
	
	\textbf{5} & & & \textbf{Produktions- und Fertigungsanforderungen} \\
	5.1 & F & & Funktionsprototyp bis zum 07.02.2020 \\
	5.2 & W & & Verwendung wasserfester Materialien \\
	
	\midrule
	
	\textbf{6} & & & \textbf{Sonstiges} \\
	6.1 & F & & \\
	
	\bottomrule
	
	\caption{Anforderungsliste (F$\,=\,$Festanforderung, W$\,=\,$Wunschanforderung)}
	\label{anforderungsliste}
\end{longtable}


\section{Funktionsstruktur}

\begin{itemize}
	\item Allgemeine kybernetische Black-Box-Darstellung
	\item Hierarchische Funktionsstruktur (min. 10 Teilfunktionen)
	\item Funktionsmodell mit Darstellung der (min. 10) wichtigsten Funktionen
\end{itemize}


\section{Lösungsprinzipien}

\begin{itemize}
	\item Morphologischer Kasten mit (jeweils min. 4) Teillösungsprinzipien (ggf. durch geeignete Lösungsfindungsmethoden) für diese wichtigsten Funktionen
	\item Verträglichkeitsmatrix für die Teilprinzipien
	\item Kennzeichnung von min. 3 möglichen Gesamtlösungskombinationen im morphologischen Kasten
	\item Bewertung dieser Gesamtlösungskombinaitonen
\end{itemize}

\section{Technische Prinzipskizze}

Anfertigung einer technischen Prinzipskizze zur Verdeutlichung des Wirkprinzips