%Dokumentklasse
\documentclass[a4paper,12pt, headings=small, bibtotoc, numbers=noenddot]{scrreprt} %bibtotoc = bibliography to table of contents, bindet Literaturverzeichnis in Inhaltsverzeichnis ein
\usepackage[left= 2.5cm,right = 2cm, top = 2cm, bottom = 2cm]{geometry} %Seitenränder
\usepackage[onehalfspacing]{setspace} %1,5 Zeilenabstand

% ============= Packages =============

% Dokumentinformationen
\usepackage[
	pdftitle={Titel der Arbeit},
	pdfsubject={Abschlussarbeit},
	pdfauthor={Max Mustermann},
	pdfkeywords={}
	pdftex=true, 
	colorlinks=true,
 	breaklinks=true,
	citecolor=black,
	linkcolor=black,	
	menucolor=black,	
	urlcolor=black,
]{hyperref}


% Standard Packages
\usepackage{libertine}	%Schriftart "Biolinum", Sans Serif
\renewcommand*\familydefault{\sfdefault} %Setzt die Standardschriftart auf Sans Serif
\usepackage[T1]{fontenc}
\usepackage{mathptmx} %Times new roman
\usepackage[utf8]{inputenc}
\usepackage[ngerman]{babel}
\AtBeginDocument{\renewcommand{\chaptername}{}}

\usepackage{rotating}
\usepackage{subfigure}
\usepackage{graphicx}
\graphicspath{{img/}}

\usepackage{multirow}

\usepackage{fancyhdr}
\usepackage{lmodern}
\usepackage{color}
\usepackage{transparent}
\usepackage[justification=centering, singlelinecheck=false]{caption} %Damit die Tabellenbeschriftung zentriert steht
\usepackage{nomencl}
\usepackage{tabularx}
\usepackage{longtable}
\usepackage{booktabs}
\usepackage{pdfpages}
\usepackage{acronym}
\usepackage{setspace}

%Schriftgröße von section auf 12pt setzen
\addtokomafont{section}{\normalsize} 
\addtokomafont{chapter}{\large}
%Abbildungsbeschriftungen fett gedruckt einstellen
\addtokomafont{captionlabel}{\bfseries}


%Abstände zwischen Text und Überschriften
%Zeilenabstände bei Unterkapiteln
\RedeclareSectionCommand[beforeskip=18pt, afterskip=6pt]{section}
%Zeilenabstände bei Unter-unterkapiteln 
\RedeclareSectionCommand[beforeskip=12pt,afterskip=1pt]{subsection}
\RedeclareSectionCommand[%
  beforeskip=0pt,
  afterskip=1\baselineskip plus .1\baselineskip minus .167\baselineskip
]{chapter}


% zusätzliche Schriftzeichen der American Mathematical Society
\usepackage{amsfonts}
\usepackage{amsmath}

\usepackage{float} %erlaubt die Verwendung von H in Bildern
% mu im normalen Text: \textmu
\usepackage{textcomp}

\usepackage[backend=biber, style=alphabetic,uniquename=allfull,minalphanames=4,maxalphanames=4]{biblatex}
\addbibresource{Literatur.bib}

% nicht einrücken nach Absatz
\setlength{\parindent}{0pt}

\setcounter{secnumdepth}{4}
\setcounter{tocdepth}{4}

%url umbrechen
\usepackage{url}

\usepackage{microtype} %Sorgt für bessere Platzausnutzung der \hbox bei Umbruechen
\setlength{\emergencystretch}{1em} %Sorgt für bessere Platzausnutzung der \hbox bei Umbruechen

\setcounter{biburlnumpenalty}{100}
\setcounter{biburlucpenalty}{100}
\setcounter{biburllcpenalty}{100}
% ============= Package Einstellungen & Sonstiges ============= 

%römische Aufzählungen mit \RM{Zahl}
\newcommand{\RM}[1]{\MakeUppercase{\romannumeral #1}}

% ============= Dokumentbeginn =============

\begin{document}

\pagestyle{empty}   %leere Seite

%Seitennummerierung neu beginnen, Zahlen [arabic], röm.Zahlen [roman,Roman], Buchstaben [alph,Alph]

%Deckblatt
\label{Coversheet}
\begin{minipage}{0.5\textwidth}
\begin{flushleft}
\includegraphics[width=\textwidth]{Bilder/LKT_logo.png}
\end{flushleft}
\end{minipage}
\begin{minipage}{0.5\textwidth}
\begin{flushright}
\includegraphics[width=0.65\textwidth]{Bilder/UdS_Logo.png}
\end{flushright}
\end{minipage}
\headrule

\vspace{1.5cm}

\begin{center}
\LARGE \textbf{Vitamintablettenspender}\\ 
\vspace{3cm}
\textbf{Projektarbeit} \\
\Large Systementwicklungsmethodik 2\\ 
WS 2019/20\\
im Studiengang Systems Engineering\\
der Naturwissenschaftlich-Technischen Fakultät \\
der Universität des Saarlandes\\
\vspace{2.5cm}
\Large von\\
\vspace{1cm}
Kristian König und Tim Goll\\
Matrikelnummern: 2560270, 2553050 \\
Gruppennummer: 1\\
\vspace{1cm}
Saarbrücken, 2019
\end{center}
\newpage 

%Leerseite für doppelseitigen Druck
\thispagestyle{empty}
\quad 
\newpage

%Aufgabenstellung
%Hier wird die der verfassten Arbeit zugehörige Aufgabenstellung im PDF-Format eingebunden. Dazu wird im folgenden Block der Titel ,,Aufgabenstellung.pdf'' durch den entsprechenden Dateinamen ersetzt.
\label{Aufgabenstellung}
\includepdf[pages={1-5}]{Bilder/aufgabenstellung}
\newpage

%Leerseite für doppelseitigen Druck
%\thispagestyle{empty}
%\quad 
%\newpage

%Eidestatliche Erklärung
%\label{Erklaerung}

%\begin{center} 
%\begin{minipage}[t]{120mm}
% \vspace{8cm}
%Ich versichere hiermit, dass ich die vorliegende Arbeit
%angegebenen Quellen und Hilfsmittel benutzt habe.\\  
%\end{minipage}
%\end{center}

%\vspace{3cm}
%Saarbrücken, den 30.04.2018  ~~~~~~~~~~~~~~~~~~~~~~~~~~~~~~~~~~~~~~~~~~~~~~~ Vorname Nachname
%\newpage

%Leerseite für doppelseitigen Druck
%\thispagestyle{empty}
%\quad 
%\newpage

\pagestyle{fancy} \newpage   %Seitenlayout normal
\fancyhf{} 
\fancyhead[L]{\nouppercase{{\fancyplain{}{\rightmark }}} } 
\fancyhead[R]{\thepage} 
\renewcommand{\headrulewidth}{0.5pt} 
\pagenumbering{Roman}


%Inhaltsverzeichnis
\tableofcontents \thispagestyle{empty} \newpage

%Seitennummerierung neu beginnen, Zahlen [arabic], röm.Zahlen [roman,Roman], Buchstaben [alph,Alph]
\pagenumbering{arabic} \newpage
\renewcommand*{\chapterpagestyle}{empty}

%Einbinden aller .tex-Dateien, welche die einzelnen Kapitel beinhalten
% !TeX encoding = UTF-8
% !TeX root = ../Vitamintablettenspender.tex

\chapter{Lastenheft}

Es ist die Entwicklung eines innovativen Haushalt-Gadgets in AM-\&Multimaterial-Design vorgegeben. Dabei soll ein bedeutsames aber zeitgleich auch revolutionäres und innovatives Produkt entworfen werden. Vom Kunde ist ein Vitamintablettenspender gewünscht. Dieser soll über folgende Funktionen verfügen:

\begin{itemize}
	\item Aufbewahrungsmöglichkeit für vier unterschiedliche Vitamintablettensorten
	\item Automatischer Tablettenauswurf in einen Trinkbehälter
	\item Automatisches Auffüllen des Trinkbehälters mit Wasser
	\item Handlich
	\item Leicht
\end{itemize}

Dazu wurden zusätzliche Kundenbefragungen und Umfragen hinsichtlich wünschenswerter Zusatzfunktionen durchgeführt. Als begeisternde Kriterien wurden folgende Wunschanforderungen formuliert:

\begin{itemize}
	\item Touchdisplay zur Bedienung
	\item Füllstandanzeige der Tablettenreservoirs
	\item großer Wasserbehälter
\end{itemize}

Das Produkt soll dem Kunden im Februar 2020 mittels einem fertigen Prototypen vorgestellt werden.
% !TeX encoding = UTF-8
% !TeX root = ../Vitamintablettenspender.tex

\chapter{Konzeptionierung}

Im Entwicklungsprozess nimmt die Suche nach der optimalen Lösung für das vom Kunden gewünschte Produkt die Hauptaufgabe ein. Das Ergebnis soll nachvollziehbar und objektiv bewertbar sein. Dafür wird im Folgenden zunächst eine umfassende Planung des Produktes hinsichtlich Markt- und Wettbewerbschancen durchgeführt.

Die Entwicklung des Konzepts erfolgt darauffolgend anhand eines Projektplans, der den vom Kunden gewünschten Termin zur Vorstellung des Produktes mit einem Prototypen berücksichtigt. In der Anforderungsliste werden die Anforderungen des Kunden aus dem Lastenheft konkretisiert und durch interne Spezifikationen ergänzt. Damit wird eine Basis zur Entwicklung von Lösungsideen geschaffen.

Hierfür werden zunächst die Zusammenhänge von Anforderungen und Funktionen abstrakt in der Funktionsstruktur dargestellt, wobei das Loslösen vom Gegenständlichen und von vorzeitigen Festlegungen auf ein bestimmtes Lösungskonzept ermöglicht wird. Zur Systematisierung der Suche und Auswahl des optimalen Lösungsprinzips wird im morphologischen Kasten alle Lösungsoptionen aller Teilfunktionen berücksichtigt und verschiedene Gesamtlösungskombinationen unter Beachtung von Konflikten untereinander gebildet. Die abgesicherte Festlegung des zu realisierenden Lösungskonzeptes erfolgt in der Nutzwertanalyse. Zuletzt wird ein Grobentwurf zur Verdeutlichung des Wirkprinzips angefertigt.

\section{Produktplanung}

Marktanalysen, Wettbewerbsanalysen, Technologieanalysen und Patentanalyse


\section{Projektplan}

Für die Gestaltungsaufgabe mit Planung von

\begin{itemize}
\item Aufgaben
\item Dauer, Beginn und Ende der Aufgaben
\item Abhängigkeiten zwischen Aufgaben
\item ggf. kritischem Pfad
\end{itemize}

\section{Anforderungsliste}

\begin{itemize}
	\item Anforderungen des Lastenheftes präzisiert sowie um sinnvolle Anforderungen mund Angaben inkl. Verweis auf Lage im Kano-Diagramm ergänzt (min. 20 Anforderungen)
	\item Konsistenzmatrix für (min. 10) Hauptanforderungen
\end{itemize}

\newcolumntype{L}[1]{>{\raggedright\arraybackslash}p{#1}} % linksbündig mit Breitenangabe
\newcolumntype{C}[1]{>{\centering\arraybackslash}p{#1}} % zentriert mit Breitenangabe
\newcolumntype{R}[1]{>{\raggedleft\arraybackslash}p{#1}} % rechtsbündig mit Breitenangabe

\begin{longtable}{C{0.05\linewidth}C{0.05\linewidth}C{0.05\linewidth}L{0.75\linewidth}}
	\toprule
 	
 	\textbf{Nr.} & \textbf{F/W} & \textbf{Gew.} & \textbf{Beschreibung und Erläuterung}  \\
	
	\toprule
	\endfirsthead
	
	\textbf{1} & & & \textbf{Funktionsanforderungen}  \\
	1.1 & F & & Wirft gewünschte Vitamintablette aus  \\
	1.2 & W & & Tabletten werden automatisch ausgeworfen, wenn Glas platziert wird \\
	1.3 & W & & Immer nur eine der Tageszeit entsprechende Tablette darf ausgeworfen werden um Vitaminbalance zu garantieren \\
	1.4 & W & & Modernes Design \\
	1.5 & F & & Gute Standfestigkeit oder Möglichkeit zu Wandmontage \\
	1.6 & F & & Anzeige der aktuellen Uhrzeit \\
	1.7 & F & & Anzeige der nächsten Tablette mit deren Uhrzeit \\
	1.8 & F & & Einfache und intuitive Bedienung \\
	1.9 & W & & Nutzerprofile, sodass zwischen mehreren Nutzern gewechselt werden kann \\
	1.10 & F & & Vorrat-Leer Sensor um Nutzer auf Röhrchenwechsel hinzuweisen \\
	1.11 & W & & Warnton, wenn Tablette vergessen wurde \\
	
	\midrule
	
	\textbf{2} & & & \textbf{Mechanische/Geometrische Anforderungen} \\
	2.1 & F & & Muss kompatibel sein mit handelsüblichen Vitaminröhrchen \\
	2.2 & F & & Verwendung robuster Sensoren um Lebensdauer zu erhöhen \\
	2.3 & W & & Handliches Gewicht: $<\,5\,\text{kg}$ \\
	2.4 & W & & Maximale Abmessungen: $X\,\text{cm}\,\times\,Y\,\text{cm}\,\times\,Z\,\text{cm}$ \\
	
	\midrule
	
	\textbf{3} & & & \textbf{Sicherheitsanforderungen} \\
	3.1 & F & & Auswurfsmechanismus muss geschützt sein, sodass Finger nicht versehentlich hinein gerät \\
	3.2 & F & & Standfeste Position des Glases \\
	
	\midrule 
	
	\textbf{4} & & & \textbf{Umwelt- und Wartungsanforderungen} \\
	4.1 & F & & Verwendung von lebensmittelechten Materialien \\
	
	\midrule
	
	\textbf{5} & & & \textbf{Produktions- und Fertigungsanforderungen} \\
	5.1 & F & & Funktionsprototyp bis zum 07.02.2020 \\
	5.2 & W & & Verwendung wasserfester Materialien \\
	
	\midrule
	
	\textbf{6} & & & \textbf{Sonstiges} \\
	6.1 & F & & \\
	
	\bottomrule
	
	\caption{Anforderungsliste (F$\,=\,$Festanforderung, W$\,=\,$Wunschanforderung)}
	\label{anforderungsliste}
\end{longtable}


\section{Funktionsstruktur}

\begin{itemize}
	\item Allgemeine kybernetische Black-Box-Darstellung
	\item Hierarchische Funktionsstruktur (min. 10 Teilfunktionen)
	\item Funktionsmodell mit Darstellung der (min. 10) wichtigsten Funktionen
\end{itemize}


\section{Lösungsprinzipien}

\begin{itemize}
	\item Morphologischer Kasten mit (jeweils min. 4) Teillösungsprinzipien (ggf. durch geeignete Lösungsfindungsmethoden) für diese wichtigsten Funktionen
	\item Verträglichkeitsmatrix für die Teilprinzipien
	\item Kennzeichnung von min. 3 möglichen Gesamtlösungskombinationen im morphologischen Kasten
	\item Bewertung dieser Gesamtlösungskombinaitonen
\end{itemize}

\section{Technische Prinzipskizze}

Anfertigung einer technischen Prinzipskizze zur Verdeutlichung des Wirkprinzips
% !TeX encoding = UTF-8
% !TeX root = ../Vitamintablettenspender.tex

\chapter{Methodische Werkstoffauswahl}
Von der Produktentwicklung werden leistungsfähige Systeme bei gleichzeitiger Erfüllung von Sicherheits- und Umweltverträglichkeitsanforderungen erwartet. Steigenden Kosten für Material und Energie werden im Zuge des Multi-Material-Designs mittels einer methodischen Werkstoffauswahl durch die \glqq Ashby-Methode\glqq{} und den dabei entstehenden intelligenten Lösungen entgegengewirkt. Dabei werden unterschiedlichste Aspekte wie die Möglichkeit zur generativen Fertigung bzw. dem Additive Manufacturing berücksichtigt. Die Auslegung der Hauptkomponenten des Produktes unterliegt den jeweils gültigen Randbedingungen und ihren zu erfüllenden Funktionen und werden hinsichtlich ihrer freien Variablen untersucht und bezüglich der Werkstoffauswahl optimiert. Eine \glqq Performance-Rechnung\grqq{} ermöglicht die Einbeziehung von mathematischen und physikalischen Formeln zur Entmystifizierung der Werkstofffaktoren. In der kommerziellen Software CES Selector der Firma Granta Design Ltd. (Cambridge, Vereinigtes Königreich), die von Michael Ashby, dem Erfinder der Ashby-Methode, gegründet wurde, wird mittels einer computergestützten Datenbank auf Basis der Performance-Rechnung die fünf besten in Frage kommenden Werkstofflösungen ermittelt, woraus die optimale Werkstofflösung durch \glqq weiche\grqq{} Faktoren resultiert.\\
Für die Anwendung der Ashby-Methode muss zunächst die Hauptfunktion des Bauteils bestimmt werden. Darauffolgend müssen Randbedingungen sog. Constraints definiert werden, auf Basis derer ein Ziel zur Optimierung gesetzt wird. Dabei treten freie Variablen auf, die während des Designprozesses des Produktes variiert werden können. Die Randbedingungen liefern im Allgemeinen numerische Gleichungen die einen Performance-Index implizieren. Ein höherer Performance-Index bedeutet konventionell ein besseres Material.\\
Im Folgenden wird das Prinzip von Ashby zur Materialauswahl auf die vier Hauptkomponenten des Systems angewandt.

\section{Standfuß und Gehäuse}
Das Gehäuse inklusive des Standfußes ist im Querschnitt zusammen mit den wesentlich angreifenden Lasten in Abb. \ref{fig:0301skizze} (a) dargestellt. Die Streckenlast $q_0$ fasst die Gewichtskräfte des Tablettenauswurfs, der Aktoren sowie der Behälter für die Lagerung der Tabletten zusammen. Die Einspannung am unteren Ende symbolisiert die Lagerung des Bauteils an der Oberfläche. Eine geeignete Abstraktion des Systems wird durch die Reduzierung der Streckenlast sowie der Einspannung auf zwei resultierende Kräfte in \ref{fig:0301skizze} (a) beschrieben. Darin ist die Kraft $F_{res}$ die Resultierende aus der Streckenlast und greift im Schwerpunkt der Last an. Die Einspannung am unteren Ende reduziert sich zu der für die Berechnung wesentlichen vertikalen Komponente $F_{L,V}$. Unter Vernachlässigung des Eigengewichts des Bauteils gilt näherungsweise $F_{res}\,\approx\,F_{L,V}$. Die Werkstoffauswahl und die Optimierung einer Zielfunktion kann für den dargestellten kritischen Teil des Gehäuse stellvertretend für die gesamte Struktur vorgenommen werden.
\begin{figure}[H]
	\centering
	\includegraphics[width=1.0\linewidth]{chapter/Bilder/0301skizze}
	\caption{(a, links) Querschnittsdarstellung des Gehäuses und angreifende Lasten, (b, rechts) vereinfachte Darstellung mit angreifenden resultierenden Kräften und der kritischen Zone}
	\label{fig:0301skizze}
\end{figure}
\begin{figure}[H]
	\centering
	\includegraphics[width=1.0\linewidth]{chapter/Bilder/0301modell}
	\caption{Abstrahiertes Modell einer Platte unter reiner Biegung mit dem Biegemoment $M_b$}
	\label{fig:0301model}
\end{figure}
Dabei ist die \textbf{Funktion} des Bauteils den Tablettenauswurf, die Aktoren sowie die Tablettenbehälter zu stützen und in der Höhe zu halten. Auf Basis der Anforderungsliste in \ref{anforderungsliste} und der benötigten mechanischen Eigenschaften können folgende \textbf{Randbedingungen/Constraints} definiert werden:
\begin{itemize}
	\item Recyclebares Material
	\item Höhe des kritischen Teils/Länge der Platte $L\,=\,180\,$mm
	\item Material muss ein guter Isolator sein, bzw. $\rho_e\,>\,10^{19}\,\mu\Omega$cm
	\item CO$_2$-Ausstoß bei der Materialgewinnung $<\,2\,\frac{kg\,(\text{CO}_2)}{kg\,(\text{Material})}$
	\item Biegesteifigkeit $S$ so, dass bei Einleitung der Kraft $F_{res}$ das an der Platte resultierende Biegemoment $M_b$ eine maximale Durchbiegung $w_{\text{max}}$ zur Folge hat
\end{itemize}
Das \textbf{Ziel} der Werkstoffauswahl ist die Reduzierung der Kosten für das Gehäuse. Als \textbf{freie Variablen} treten dabei zum einen die Querschnittfläche der rückwandigen Platte, welche durch Konzeptleichtbau im Verlauf des Produktdesigns optimiert werden kann, zum anderen die Wahl des Materials.
Dadurch, dass die Biegesteifigkeit $S$ sowie die Höhe der Platte $L$ vorgegeben sind, ergeben sich die beiden Gleichungen als Grundlage der Performance-Rechnung. Hierbei gilt zunächst für die Durchbiegung in der vertikalen Mitte der Platte ausgehend von der Differentialgleichung 2. Ordnung:
\begin{equation}
	E\,I\,w''(x)\,=\,-\,M_b\,
\end{equation}
durch zweifaches Integrieren, Umstellen und Einsetzen unter Vernachlässigung des Vorzeichens, das ausschließlich für die Richtung der Durchbiegung berücksichtigt werden muss:
\begin{equation} \label{durchbiegung}
	w\,\left(\frac{L}{2}\right)\,=\,\frac{M_b \,L^2}{8\,E\,I}\,\le\,w_{\text{max}},
\end{equation}
wobei $E\,I$ der Biegesteifigkeit $S$ entspricht. Als weitere Gleichung ergibt sich der Zusammenhang für die effektiven Kosten $K$, die in erster Linie über den Kostenfaktor $C_m$ mit der Masse der Platte korrelieren. Unter Berücksichtigung der Geometrie und der Materialdichte $\rho$ ergibt sich:
\begin{equation} \label{kosten}
	K\,=\,C_m\,m\,=\,C_m\,\rho\,A\,L\,.
\end{equation}
Dabei ist das Flächenträgheitsmoment eine Funktion der Querschnittfläche, die als rechteckig angenähert werden kann. Hier sind jedoch die Kantenlängen $h$ und $b$ variabel. Damit gilt für die Querschnittsfläche:
\begin{equation} \label{flächeninhalt}
A\,=\,h\,b
\end{equation}
sowie für das Flächenträgheitsmoment:
\begin{equation} \label{flächenträgheitsmoment}
I\,=\,\frac{h\,b^3}{12}\,.
\end{equation}
Einsetzen von \ref{flächenträgheitsmoment} in \ref{durchbiegung} und \ref{flächeninhalt} in \ref{kosten} liefert die Gleichungen:
\begin{equation} \label{eingesetzt}
	w_{\text{max}}\,\ge\,\frac{3\,M_b \,L^2}{2\,E\,h\,b^3}\,, \qquad
	K\,=\,C_m\,\rho\,h\,b\,L
\end{equation}
Eliminieren von $b$ führt in \ref{eingesetzt} schließlich zu
\begin{equation}\label{performance}
K\,=\,\underbrace{\left(\frac{3\,M_b\,L^5\,h^3}{2\,w_{\text{max}}}\right)^{\frac{1}{3}}}_\text{vorgegeben, bzw. geometrieabhängig}\,\frac{C_m\,\rho}{E^{\frac{1}{3}}}\,.
\end{equation}
Damit ist die \textbf{Zielfunktion} mit
\begin{equation} \label{zielfkt}
P_{\text{CR}}\,=\,\frac{1}{K}\,=\,\frac{E^\frac{1}{3}}{C_m\,\rho}
\end{equation}
definiert, wobei $P_{\text{CR}}$ zu maximieren ist.\\
CES-materialselection ...\\

\section{Trichterförmige Tablettenrutsche}
Die Funktion

\begin{itemize}
	\item Material muss generativ bzw. additiv fertigbar sein
	\item CO$_2$-Ausstoß bei der Materialgewinnung $<\,2\,\frac{kg\,(\text{CO}_2)}{kg\,(\text{Material})}$
	\item Material muss recyclebar sein
	\item 
\end{itemize}
Maximiere spezifische Festigkeit für 

\section{Tablettenlagerung}
\begin{itemize}
	\item Durchsichtiges/transparentes Material
	\item 
\end{itemize}
Minimiere CO2 für ausreichende Biegesteifigkeit
% !TeX encoding = UTF-8
% !TeX root = ../Vitamintablettenspender.tex

\chapter{Ausarbeitung}
\include{chapter/05_Prototypenbau}

%Abbildungsverzeichnis
\listoffigures \thispagestyle{empty} \newpage

%Tabellenverzeichnis
\listoftables \thispagestyle{empty} \newpage

\printbibliography

%Einbinden des Anhangs
\appendix
\chapter{Anhang}
\label{Anhang}
Hier könnte Ihr Anhang stehen!

\end{document}