\chapter{Konzeptionierung}
Im Entwicklungsprozess nimmt die Suche nach der optimalen Lösung für das vom Kunden gewünschte Produkt die Hauptaufgabe ein. Das Ergebnis soll nachvollziehbar und objektiv bewertbar sein. Dafür wird im Folgenden zunächst eine umfassende Planung des Produktes hinsichtlich Markt- und Wettbewerbschancen durchgeführt. Die Entwicklung des Konzepts erfolgt darauffolgend anhand eines Projektplans, der den vom Kunden gewünschten Termin zur Vorstellung des Produktes mit einem Prototypen berücksichtigt. In der Anforderungsliste werden die Anforderungen des Kunden aus dem Lastenheft konkretisiert und durch interne Spezifikationen ergänzt. Damit wird eine Basis zur Entwicklung von Lösungsideen geschaffen. Hierfür werden zunächst die Zusammenhänge von Anforderungen und Funktionen abstrakt in der Funktionsstruktur dargestellt, wobei das Loslösen vom Gegenständlichen und von vorzeitigen Festlegungen auf ein bestimmtes Lösungskonzept ermöglicht wird. Zur Systematisierung der Suche und Auswahl des optimalen Lösungsprinzips wird im morphologischen Kasten alle Lösungsoptionen aller Teilfunktionen berücksichtigt und verschiedene Gesamtlösungskombinationen unter Beachtung von Konflikten untereinander gebildet. Die abgesicherte Festlegung des zu realisierenden Lösungskonzeptes erfolgt in der Nutzwertanalyse. Zuletzt wird ein Grobentwurf zur Verdeutlichung des Wirkprinzips angefertigt.

\section{Produktplanung}
Marktanalysen, Wettbewerbsanalysen, Technologieanalysen und Patentanalyse

\section{Projektplan}
für die Gestaltungsaufgabe mit Planung von
\begin{itemize}
\item Aufgaben
\item Dauer, Beginn und Ende der Aufgaben
\item Abhängigkeiten zwischen Aufgaben
\item ggf. kritischem Pfad
\end{itemize}

\section{Anforderungsliste}
\begin{itemize}
\item Anforderungen des Lastenheftes präzisiert sowie um sinnvolle Anforderungen mund Angaben inkl. Verweis auf Lage im Kano-Diagramm ergänzt (min. 20 Anforderungen)
\item Konsistenzmatrix für (min. 10) Hauptanforderungen
\end{itemize}

\newcolumntype{L}[1]{>{\raggedright\arraybackslash}p{#1}} % linksbündig mit Breitenangabe
\newcolumntype{C}[1]{>{\centering\arraybackslash}p{#1}} % zentriert mit Breitenangabe
\newcolumntype{R}[1]{>{\raggedleft\arraybackslash}p{#1}} % rechtsbündig mit Breitenangabe
\begin{longtable}{C{0.05\linewidth}C{0.05\linewidth}C{0.05\linewidth}L{0.75\linewidth}}
	\toprule
 	\textbf{Nr.} & \textbf{F/W} & \textbf{Gew.} & \textbf{Beschreibung und Erläuterung}  \\
	\toprule
	\endfirsthead
	\textbf{1} & & & \textbf{Funktionsanforderungen}  \\
	1.1 & F & & \\
	1.2 & F & & \\
	1.3 & F & & \\
	1.4 & F & & \\
	\midrule
	\textbf{2} & & & \textbf{Mechanische/Geometrische Anforderungen} \\
	2.1 & F & & \\
	2.2 & W & & Handliches Gewicht: $<\,5\,\text{kg}$ \\
	2.3 & W & & Maximale Abmessungen: $X\,\text{cm}\,\times\,Y\,\text{cm}\,\times\,Z\,\text{cm}$ \\
	\midrule
	\textbf{3} & & & \textbf{Sicherheitsanforderungen} \\
	3.1 & F & & \\
	3.2 & F & & \\
	3.3 & F & & \\
	\midrule 
	\textbf{4} & & & \textbf{Umwelt- und Wartungsanforderungen} \\
	4.1 & F & & \\
	4.2 & W & & \\
	4.3 & W & & \\
	\midrule
	\textbf{5} & & & \textbf{Produktions- und Fertigungsanforderungen} \\
	5.1 & F & & \\
	5.2 & W & & \\
	5.3 & F & & \\
	\midrule
	\textbf{6} & & & \textbf{Sonstiges} \\
	6.1 & F & & \\
	6.2 & W & & \\
	\bottomrule
	\caption{Anforderungsliste (F$\,=\,$Festanforderung, W$\,=\,$Wunschanforderung)}
	\label{anforderungsliste}
\end{longtable}

\section{Funktionsstruktur}
\begin{itemize}
\item Allgemeine kybernetische Black-Box-Darstellung
\item Hierarchische Funktionsstruktur (min. 10 Teilfunktionen)
\item Funktionsmodell mit Darstellung der (min. 10) wichtigsten Funktionen
\end{itemize}

\section{Lösungsprinzipien}
\begin{itemize}
\item Morphologischer Kasten mit (jeweils min. 4) Teillösungsprinzipien (ggf. durch geeignete Lösungsfindungsmethoden) für diese wichtigsten Funktionen
\item Verträglichkeitsmatrix für die Teilprinzipien
\item Kennzeichnung von min. 3 möglichen Gesamtlösungskombinationen im morphologischen Kasten
\item Bewertung dieser Gesamtlösungskombinaitonen
\end{itemize}

\section{Technische Prinzipskizze}
Anfertigung einer technischen Prinzipskizze zur Verdeutlichung des Wirkprinzips